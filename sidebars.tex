\documentclass[11pt, oneside]{article}   	% use "amsart" instead of "article" for AMSLaTeX format
\usepackage{geometry}                		% See geometry.pdf to learn the layout options. There are lots.
\geometry{letterpaper}                   		% ... or a4paper or a5paper or ... 
%\geometry{landscape}                		% Activate for rotated page geometry
%\usepackage[parfill]{parskip}    		% Activate to begin paragraphs with an empty line rather than an indent
\usepackage{graphicx}				% Use pdf, png, jpg, or eps§ with pdflatex; use eps in DVI mode
								% TeX will automatically convert eps --> pdf in pdflatex		
\usepackage{amssymb}

%SetFonts

%SetFonts


\title{Brief Article}
\author{The Author}
%\date{}							% Activate to display a given date or no date

\begin{document}
\maketitle

Frequenza Il numero di volte in cui un certo valore si verifica in un insieme di dati.
Tabella delle frequenze Una tabella che rappresenta, per un certo insieme di dati,
ciascun valore distinto vicino alla sua frequenza.
Grafico a bastoncini Un grafico ottenuto da una tabella delle frequenze. L’ascissa
indica un valore, mentre la frequenza corrispondente è indicata dall’altezza di un segmento verticale.
Grafico a barre Simile a un grafico a bastoncini, tranne che in questo caso la frequenza di un valore è rappresentata dall’altezza di una barra.
Grafico poligonale Un grafico dei valori distinti e della loro frequenza in cui i punti
sono collegati da segmenti di retta.
Insieme di dati simmetrico Un insieme di dati si dice simmetrico intorno a un certo
valore x0 se le frequenze dei valori x0–c e x0+c sono le stesse per ogni scelta di c.
Frequenza relativa La frequenza del valore di un dato divisa per il numero totale di
elementi in un insieme di dati.
Grafico a torta Un grafico che indica le frequenze relative suddividendo un cerchio
in settori separati.
Istogramma Un grafico in cui i dati sono suddivisi in classi, le cui frequenze sono
rappresentate da un grafico a barre.
Istogramma delle frequenze relative Un istogramma che rappresenta le frequenze
relative dei valori nell’insieme di dati.
Diagramma ramo-foglia Simile all’istogramma, tranne per il fatto che la frequenza è
indicata allineando una dopo l’altra le ultime cifre (dette foglie) dei dati.
Diagramma di dispersione Un diagramma a due dimensioni per un insieme di dati
costituito da coppie di valori.
%\section{}
%\subsection{}



\end{document}  
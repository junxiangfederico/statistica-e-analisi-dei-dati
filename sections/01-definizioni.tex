%% Copyright (C) 2021 Alessandro Clerici Lorenzini
%
% This work may be distributed and/or modified under the
% conditions of the LaTeX Project Public License, either version 1.3
% of this license or (at your option) any later version.
% The latest version of this license is in
%   http://www.latex-project.org/lppl.txt
% and version 1.3 or later is part of all distributions of LaTeX
% version 2005/12/01 or later.
%
% This work has the LPPL maintenance status `maintained'.
%
% The Current Maintainer of this work is Alessandro Clerici Lorenzini
%
% This work consists of the files listed in work.txt


\section{Definizioni}
La definizione assiomatica della probabilità riassume in pochi punti quei concetti che si possono definire intuitivi nel calcolo delle probabilità.

\subsection*{Un'introduzione generale}
\begin{itemize}
\item Popolazione: insieme di tutti gli elementi che ci interessano
\item Campione: sottoinsieme (rappresentativo) della popolazione che viene studiato\\
		- Un campione casuale è considerato se i membri sono i scelti in modo tale che tutte le possibili scelte dei $k$ membri siano equiprobabili\\
		- Un campione casuale è chiamato stratificato se sono necessarie più informazioni iniziali
\item Frequenza assoluta ($f$): numero di occorrenza di un dato valore in un esperimento
\item Frequenza relativa: $\frac{f}{n}$ ove $f$ rappresenta la frequenza, ed $n$ il numero totale
\item Tabelle e grafici: pochi valori distinti alcuni esempi sono il grafico a bastoncini, il grafico poligonale, il grafico a barre
\item Istogrammi: tanti dati, li suddivido in range distinti, alcuni esempi sono l'istogramma delle frequenze e l'istogramma delle frequenze relative
\end{itemize}
Una serie di analisi preliminari che possiamo fare su un campione sono la \emph{media, mediana, e la moda}, che chiameremo \emph{campionaria}. 

Riguardo al nostro campione diciamo che $n$ rappresenta la dimensione, o taglia del campione; mentre ${\{x_1, ..., x_n\}}$ rappresentano gli insiemi del nostro campione.
\begin{itemize}
\item La media campionaria rappresenta la media aritmetica, e rappresenta l'operatore lineare. %Manca la definizione, prendi da pagina 1, nozione di frequenza relativa, 3 modalità totali per il calcolo della media campionaria
\item La mediana campionaria rappresenta una stima \emph{più robusta} della media campionaria (esempio patente), l'outlier o valore fuori scala, rappresenta un valore del campione che pesa in maniera negativa sulla nostra stima. 

La mediana campionaria è uno stimatore "robusto" rispetto agli outlier; viene calcolato riordinando i valori in ordine di grandezza, e prendendo il valore/valori centrali. %Guardare su foglio 1R, definizione, calcolo
\item La moda campionaria rappresenta il valore con frequenza massima
\end{itemize}


\subsection{L'algebra degli eventi}
\begin{defin}
	Scelto un insieme $\Omega$ detto spazio campionario o degli eventi, si dice esito un elemento $\omega\in\Omega$ dell'insieme ed evento un suo sottoinsieme $E\subseteq\Omega$.

\end{defin}
\begin{defin}
	Sia $\A\in2^\Omega$ una collezione di sottoinsiemi di $\Omega$. Allora $\A$ è un'algebra se
	\begin{align*}
		 & \bullet ~ \Omega\in \A                                                                                            \\
		 & \bullet ~ \forall E\subseteq\Omega\qquad E\in\A\Rightarrow \bar E\in\A \qquad\text{con }\bar E:=\Omega\setminus E \\
		 & \bullet ~ \forall E_1,\dots E_n\in\Omega\qquad \forall i~E_i\in\A\Rightarrow \bigcup\limits_{i=1}^n E_i \in\A
	\end{align*}
	$\A$ è una $\sigma$-algebra su $\Omega$ se l'ultima condizione si può estendere a unioni numerabili qualsiasi.
\end{defin}




\subsection{Assiomi di Kolmogorov}
La probabilità viene definita come una funzione di un'algebra degli eventi $\A$ in $\R$:
\begin{equation*}
	P:\A\to\R
\end{equation*}

I seguenti assiomi, detti di Kolmogorov, decretano le proprietà che la probabilità rispetta:


\subsubsection{Primo assioma}
L'immagine di $P$ è l'insieme $[0,1]\in\R$. Equivalentemente, la probabilità di qualunque evento è compresa tra $0$ e $1$:
\begin{equation*}
	\forall E\in\A\qquad 0\leq P(E)\leq 1
\end{equation*}


\subsubsection{Secondo assioma}
La probabilità dello spazio campionario è $1$:
\begin{equation*}
	P(\Omega)=1
\end{equation*}


\subsubsection{Terzo assioma}
La probabilità dell'unione di eventi mutuamente esclusivi, cioè disgiunti (intuitivamente, il cui avvenire dell'uno esclude l'avvenire dell'altro), è uguale alla somma delle probabilità dei singoli:
\begin{equation*}
	\forall E_1,\dots,E_n\in\A\qquad \forall i,j~E_i\cap E_j = \emptyset \Rightarrow P\left( \bigcup_{i=1}^n E_i \right)=\sum_{i=1}^n P(E_i)
\end{equation*}



\subsection{Teoremi elementari}
Dagli assiomi di Kolmogorov derivano alcune proprietà elementari facilmente dimostrabili.


\subsubsection{Probabilità dell'evento complementare}
\begin{defin}
	Dato un evento $E\in\A$, l'evento complementare è l'evento $\bar E := \Omega\setminus E$.
\end{defin}
\begin{teor}[probabilità dell'evento complementare] \label{t:probcompl}
	Dato un evento $E$, se la probabilità di $E$ è $P(E)$, la probabilità dell'evento complementare di $E$ è $1-P(E)$:
	\begin{equation*}
		\forall E\in\A\qquad P(\bar E)=1-P(E)
	\end{equation*}
\end{teor}
\begin{proof}
	\begin{align*}
		  & \left.
		\begin{array}{cc}
			E\cap\bar E=\emptyset \\
			E\cup\bar E=\Omega
		\end{array} \right\}  \bc{definizione di evento complementare} \\
		1 & = P(\Omega)        \bc{secondo assioma}                                  \\
		  & = P(E\cup\bar E)                                                         \\
		  & = P(E)+P(\bar E)   \bc{terzo assioma}
	\end{align*}
	ergo:
	\begin{equation*}
		P(\bar E)=1-P(E) \qedhere
	\end{equation*}
\end{proof}

\subsubsection{Probabilità dell'unione}
\begin{teor} \label{t:probunion}
	Dati due eventi $E,F\in\A$, la probabilità della loro unione è uguale alla somma delle loro probabilità meno la probabilità dell'intersezione:
	\begin{equation*}
		\forall E,F \in\A\qquad P(E\cup F)=P(E)+P(F)-P(E\cap F)
	\end{equation*}
\end{teor}
\begin{proof}
	L'unione degli eventi è scrivibile come l'unione di due insiemi disgiunti:
	\begin{equation*}
		E\cup F=E \cup (\bar E\cap F) \\[1ex]
	\end{equation*}

	Passando alle probabilità:
	\begin{align*}
		P(E\cup F) & = P(E)+P(\bar E\cap F)                                               \\
		           & = P(E)+P(\bar E\cap F)+P(E\cap F)-P(E\cap F)                         \\
		           & = P(E)+P((\bar E\cap F)\cup (E\cap F))-P(E\cap F) \bc{terzo assioma} \\
		           & = P(E)+P(F)-P(E\cap F)                            \qedbc
	\end{align*}
\end{proof}

\subsubsection{Probabilità dell'evento vuoto}
La probabilità dell'evento vuoto ($\emptyset$) è $0$. Un evento con probabilità nulla viene detto evento impossibile.
\begin{equation*}
	P(\emptyset)=0
\end{equation*}

\begin{proof}
	\begin{align*}
		P(\Omega)    & = 1 \bc{secondo assioma}                       \\
		P(\emptyset) & = P(\bar\Omega)                                \\
		             & = 1 - P(\Omega) \bc{teorema \ref{t:probcompl}} \\
		             & = 1 - 1 = 0     \qedbc
	\end{align*}
\end{proof}



\subsection{Spazi di probabilità}
\begin{defin}
	Uno spazio di probabilità è una tripla $(\Omega,\A,P)$ composta da uno spazio campionario $\Omega$, un'algebra degli eventi $\A$ e una funzione di probabilità $P$.
\end{defin}


\subsubsection{Spazi equiprobabili}
\begin{defin}
	Uno spazio è equiprobabile se gli eventi elementari (cioè corrispondenti a singoletti) hanno probabilità costante $p$.
\end{defin}

Un evento elementare è composto da un singolo esito, cioè è un singoletto dell'insieme delle parti dello spazio campionario.
\begin{teor}
	In uno spazio equiprobabile, la probabilità di ogni evento elementare $e$ è uguale al reciproco del numero $n$ degli eventi elementari (che sono ovviamente a due a due disgiunti):
	\begin{equation}
		P(e)=\frac{1}{n}
	\end{equation}
\end{teor}
\begin{proof}
	\begin{align*}
		1=P(\Omega)=P\left( \bigcup_{i=1}^n e_i \right) \bc{secondo assioma} \\
		=\sum_{i=1}^n P(e_i) = np \bc{terzo assioma}                         \\
	\end{align*}
	Ergo:
	\begin{equation*}
		\forall e_i ~ P(e_i)=p=\frac{1}{n} \qedhere
	\end{equation*}
\end{proof}

Gli eventi non elementari possono essere espressi come unione di eventi elementari.
\begin{teor}
	In uno spazio equiprobabile, dato un evento $E=\{e_1,\dots,e_k\}$:
	\begin{equation}
		P(E)=\frac{|E|}{n}
	\end{equation}
\end{teor}
\begin{proof}
	\begin{equation*}
		P(E)= \sum_{i=1}^{|E|} P(e_i) = \frac{|E|}{n} \qedhere
	\end{equation*}
\end{proof}

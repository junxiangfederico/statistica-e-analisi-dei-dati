%% Copyright (C) 2021 Alessandro Clerici Lorenzini
%
% This work may be distributed and/or modified under the
% conditions of the LaTeX Project Public License, either version 1.3
% of this license or (at your option) any later version.
% The latest version of this license is in
%   http://www.latex-project.org/lppl.txt
% and version 1.3 or later is part of all distributions of LaTeX
% version 2005/12/01 or later.
%
% This work has the LPPL maintenance status `maintained'.
%
% The Current Maintainer of this work is Alessandro Clerici Lorenzini
%
% This work consists of the files listed in work.txt

\subsection*{La statistica}
La statistica affrontata in questo corso è suddivisibile in tre mega-aree: la statistica descrittiva, che si occupa di descrivere e riassumere i dati e la statistica inferenziale, che si occupa di trarre conclusioni dai dati. Vengono affrontati i concetti di algebra degli eventi, probabilità condizionata, variabili aleatorie discrete e continue e modelli di distribuzione.

\subsection*{Notazione}
Alcune considerazioni sulle convenzioni di notazione scelte:
\begin{itemize}
	\item per il valore atteso di $X$ si usa $\ev{X}$, e non $\mathcal{E}(X)$.
	\item $P(A,B):=P(A\land B)$ o talvolta $P(A,B):=P(A\cap B)$
\end{itemize}


%% Copyright (C) 2021 Alessandro Clerici Lorenzini
%
% This work may be distributed and/or modified under the
% conditions of the LaTeX Project Public License, either version 1.3
% of this license or (at your option) any later version.
% The latest version of this license is in
%   http://www.latex-project.org/lppl.txt
% and version 1.3 or later is part of all distributions of LaTeX
% version 2005/12/01 or later.
%
% This work has the LPPL maintenance status `maintained'.
%
% The Current Maintainer of this work is Alessandro Clerici Lorenzini
%
% This work consists of the files listed in work.txt


\section{Variabili aleatorie continue}
\begin{defin}[variabile aleatoria continua]
	Una variabile aleatoria $X$ è continua se e solo se non è discreta, ossia il suo supporto non è numerabile.
\end{defin}

\begin{defin}[funzione di densità di probabilità]
	Data una variabile aleatoria $X$, la funzione di densità di probabilità di $X$ è una funzione $f_X:\R\to\R^+$ integrabile e tale che:
	\begin{equation*}
		\int_{-\infty}^{+\infty} f_X(x)dx = P(X \in \R) = 1
	\end{equation*}
\end{defin}

\begin{defin}[probabilità in estremi degeneri]
La probabilità di osservare una specifica osservazione è come nelle variabili aleatore discrete:
\begin{equation*}
	P(X = a) = \int_{a}^ {a} f_x (x) dx = 0
\end{equation*}
%Quando ragioniamo in termini di una variabile aleatore continua non ha senso chiedersi il valore di un intervallo con estremi medesimi, ma solo di intervalli con estremi non degeneri.
Per variabili aleatorie continue non ha senso calcolare la probabilità che una variabile assuma uno specifico valore, infatti:
\begin{equation*}
	\forall a\in\R \qquad P(X=a)=\int_a^a f_X(x)dx=0
\end{equation*}

In un certo intervallo che approssima $a$, ossia $[a-\frac{\varepsilon}{2},a+\frac{\varepsilon}{2}]$, invece:
\begin{equation*}
	P\left(a - \frac{\varepsilon}{2}\leq X\leq a + \frac{\varepsilon}{2}\right) = \int_{a - \frac{\varepsilon}{2}}^{a + \frac{\varepsilon}{2}} f_X(x)dx \approx \varepsilon f_X(a)
\end{equation*}

Calcolando l'integrale della funzione di densità in un intervallo del tipo $(-\infty,a)$ si calcola di fatto il valore della funzione di ripartizione in $a$:
\begin{equation*}
	\int_{-\infty}^a f_X(x)dx = P(X\leq a) = F_X(a) \\
\end{equation*}

%Possiamo riscrivere l'uguaglianza nel seguente modo:
%\begin{equation*}
%	F_X(x) = P(X \leq x) = \int_{-\infty}^x f_x(y) dy
%\end{equation*}


Per il teorema fondamentale del calcolo integrale, da ciò deriva che la funzione di densità di probabilità è la derivata della funzione di ripartizione:
\begin{equation*}
	\frac{d}{dx}F_X(x)=f(x)
\end{equation*}
\end{defin}





\begin{defin}[funzione di densità di probabilità]
La probabilità di variabili aleatorie continue viene calcolata integrando la funzione di densità di probabilità:
\begin{equation*}
	\forall B\subseteq\R, P(X\in B)=\int_B f_X(x)dx
\end{equation*}
O, in un intervallo $[a,b]$:
\begin{equation*}
	P(a\leq X\leq b)=\int_a^b f_X(x)dx
\end{equation*}
che è equivalente a quella dell'intervallo $(a,b)$.

La proprietà della definizione non è altro che la conseguenza del primo assioma di Kolmogorov esteso al continuo:
\begin{equation*}
	P(X\in\R) = \int_{-\infty}^{+\infty} f_X(x)dx = \int_\R f_X(x)dx = 1
\end{equation*}
\end{defin}


\subsection{Valore atteso}
\begin{defin}
	Il valore atteso per variabili aleatorie continue è così definito:
	\begin{equation*}
		\ev{X}=\int_{-\infty}^{+\infty} x f_X(x)dx
	\end{equation*}
\end{defin}

La varianza invece è così definita:
\begin{equation*}
	\var(x) = \ev{((x- \mu)^2) }= \int_{-\infty}^{+\infty} (x-\mu)^2 f_x(x)dx
\end{equation*}

\begin{prop} \label{prop:valatnonneg}
	Per variabili aleatorie che non assumono specificazioni negative:
	\begin{equation*}
		\forall x<0 \qquad f_X(x)=0
	\end{equation*}
	vale
	\begin{equation*}
		\int_0^{+\infty} 1-F_X(x)dx = \ev{X}
	\end{equation*}
\end{prop}

\subsection{Disuguaglianza di Markov}
Questa disuguaglianza permette di stabilire un limite superiore al valore di probabilità dalla sola conoscenza del valore atteso $\ev{x}$ a condizione che la variabile casuale sia definita non negativa.
\begin{teor}[Disuguaglianza di Markov] \label{teor:markov}
	Sia $X$ una variabile aleatoria non negativa. Allora per ogni $a$ reale positivo:
	\begin{equation*}
		P(X\geq a) \leq \frac{\ev{X}}{a}
	\end{equation*}
\end{teor}
\begin{proof}
	\begin{align*}
		\ev{X} & = \int_{-\infty}^{+\infty} x f_X(x) dx                                      \\
		       & = \int_0^{+\infty} x f_X(x) dx                  \bc{variabile non negativa} \\
		       & = \underbrace{\int_0^a x f_X(x) dx}_{\geq 0} + \underbrace{\int_a^{+\infty} x f_X(x) dx}_{x \geq a} \\
		       & \geq \int_a^{+\infty} x f_X(x) dx                                           \\
		       & \geq \int_a^{+\infty} a f_X(x) dx                  \bc{in quanto $x\geq a$} \\
		       & = a \int_a^{+\infty} f_X(x) dx                                              \\
		       & = a P(X\geq a)
	\end{align*}
	ergo
	\begin{equation*}
		\frac{\ev{X}}{a}\geq P(X\geq a)
	\end{equation*}
	da ciò possiamo anche dire:
	\begin{equation*}
		P(X < a) = 1 - P(X \geq a) \geq 1- \frac{\ev{x}}{a}
		\end{equation*}
\end{proof}



\subsection{Disuguaglianza di Tchebyshev}
La disuguaglianza di Tchebyshev dà una limitazione superiore alla probabilità che l'esito di una variabile aleatoria si discosti dal suo valore atteso di una quantità maggiore o uguale a una soglia scelta.
\begin{teor}[Disuguaglianza di Tchebyshev]
	Sia $X$ una variabile aleatoria, con $\ev{X}=\mu, \var(X)=\sigma^2$. Allora:
	\begin{equation*}
		\forall r>0\qquad P(\abs{X-\mu}\geq r)\leq \frac{\sigma^2}{r^2}
	\end{equation*}
\end{teor}
\begin{proof}
	\begin{equation*}
		\abs{X-\mu}\geq r \Leftrightarrow (X-\mu)^2 \geq r^2
	\end{equation*}
	Passando alle probabilità:
	\begin{align*}
		P(\abs{X-\mu}\geq r) & = P((X-\mu)^2 \geq r^2)                                       \\
		                     & \leq \frac{\ev{(X-\mu)^2}}{r^2} \bc{per disuguaglianza di Markov} \\
		                     & =\frac{\var(X)}{r^2} 					\\
		                     & = \frac{\ev{(x-\mu)^2}}{r^2} = \frac{\sigma^2}{r^2}
	\end{align*}
	Ergo:
	\begin{equation*}
		P(\abs{X-\mu}\geq r)\leq \frac{\sigma^2}{r^2}
	\end{equation*}
\end{proof}

Un'interessante applicazione della disuguaglianza di Tchebyshev è quella che riguarda la deviazione standard, che esprime l'andamento della probabilità allontanandosi dal valore atteso di quantità ripetute della deviazione standard:
\begin{equation*}
	P(\abs{X-\mu}\geq k\sigma)\leq\frac{1}{k^2}
\end{equation*}
